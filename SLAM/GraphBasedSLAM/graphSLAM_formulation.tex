\documentclass{article}

\usepackage{amsfonts}
\usepackage{amsmath,amssymb,amsfonts}
\usepackage{textcomp}
\usepackage{fullpage}
\usepackage{setspace}
\usepackage{float}
\usepackage{cite}
\usepackage{graphicx}
\usepackage{caption}
\usepackage{subcaption}
\usepackage[pdfborder={0 0 0}, pdfpagemode=UseNone, pdfstartview=FitH]{hyperref}

\DeclareMathOperator*{\argmax}{arg\,max}
\DeclareMathOperator*{\argmin}{arg\,min}

\def\keyterm{\textit}

\newcommand{\transp}{{\scriptstyle{\mathsf{T}}}}



\begin{document}

\title{Graph SLAM Formulation}
\author{Jeff Irion}
\date{}

\maketitle
\vspace{3em}


\section{Problem Formulation}

Let a robot's trajectory through its environment be represented by a sequence of $N$ poses: $\mathbf{p}_1, \mathbf{p}_2, \ldots, \mathbf{p}_N$.  Each pose lies on a manifold: $\mathbf{p}_i \in \mathcal{M}$.  Simple examples of manifolds used in Graph SLAM include 1-D, 2-D, and 3-D space, i.e., $\mathbb{R}$, $\mathbb{R}^2$, and $\mathbb{R}^3$.  These environments are \keyterm{rectilinear}, meaning that there is no concept of orientation.  By contrast, in $SE(2)$ problem settings a robot's pose consists of its location in $\mathbb{R}^2$ and its orientation $\theta$.  Similarly, in $SE(3)$ a robot's pose consists of its location in $\mathbb{R}^3$ and its orientation, which can be represented via Euler angles, quaternions, or $SO(3)$ rotation matrices.  

As the robot explores its environment, it collects a set of $M$ measurements $\mathcal{Z} = \{\mathbf{z}_j\}$.  Examples of such measurements include odometry, GPS, and IMU data.  Given a set of poses $\mathbf{p}_1, \ldots, \mathbf{p}_N$, we can compute the estimated measurement $\hat{\mathbf{z}}_j(\mathbf{p}_1, \ldots, \mathbf{p}_N)$.  We can then compute the \keyterm{residual} $\mathbf{e}_j(\mathbf{z}_j, \hat{\mathbf{z}}_j)$ for measurement $j$.  The formula for the residual depends on the type of measurement.  As an example, let $\mathbf{z}_1$ be an odometry measurement that was collected when the robot traveled from $\mathbf{p}_1$ to $\mathbf{p}_2$.  The expected measurement and the residual are computed as
%
\begin{align*}
    \hat{\mathbf{z}}_1(\mathbf{p}_1, \mathbf{p}_2) &= \mathbf{p}_2 \ominus \mathbf{p}_1 \\
    \mathbf{e}_1(\mathbf{z}_1, \hat{\mathbf{z}}_1) &= \mathbf{z}_1 \ominus \hat{\mathbf{z}}_1 = \mathbf{z}_1 \ominus (\mathbf{p}_2 \ominus \mathbf{p}_1),
\end{align*}
%
where the $\ominus$ operator indicates inverse pose composition.  We model measurement $\mathbf{z}_j$ as having independent Gaussian noise with zero mean and covariance matrix $\Omega_j^{-1}$; we refer to $\Omega_j$ as the \keyterm{information matrix} for measurement $j$.  That is,
\begin{equation}
    p(\mathbf{z}_j \ | \ \mathbf{p}_1, \ldots, \mathbf{p}_N) = \eta_j \exp \left( (-\mathbf{e}_j(\mathbf{z}_j, \hat{\mathbf{z}}_j))^\transp \Omega_j \mathbf{e}_j(\mathbf{z}_j, \hat{\mathbf{z}}_j) \right), \label{eq:observation_probability}
\end{equation}
where $\eta_j$ is the normalization constant.

The objective of Graph SLAM is to find the maximum likelihood set of poses given the measurements $\mathcal{Z} = \{\mathbf{z}_j\}$; in other words, we want to find 
%
\begin{equation*}
    \argmax_{\mathbf{p}_1, \ldots, \mathbf{p}_N} \ p(\mathbf{p}_1, \ldots, \mathbf{p}_N \ | \ \mathcal{Z}) 
\end{equation*}
%
Using Bayes' rule, we can write this probability as
%
\begin{align}
    p(\mathbf{p}_1, \ldots, \mathbf{p}_N \ | \ \mathcal{Z}) &= \frac{p( \mathcal{Z} \ | \ \mathbf{p}_1, \ldots, \mathbf{p}_N) p(\mathbf{p}_1, \ldots, \mathbf{p}_N) }{ p(\mathcal{Z}) } \notag \\
    &\propto p( \mathcal{Z} \ | \ \mathbf{p}_1, \ldots, \mathbf{p}_N), \label{eq:bayes}
\end{align}
%
since $p(\mathcal{Z})$ is a constant (albeit, an unknown constant) and we assume that $p(\mathbf{p}_1, \ldots, \mathbf{p}_N)$ is uniformly distributed.  Therefore, we can use \eqref{eq:observation_probability} and \eqref{eq:bayes} to simplify the Graph SLAM optimization as follows:
%
\begin{align*}
    \argmax_{\mathbf{p}_1, \ldots, \mathbf{p}_N} \ p(\mathbf{p}_1, \ldots, \mathbf{p}_N \ | \ \mathcal{Z}) &= \argmax_{\mathbf{p}_1, \ldots, \mathbf{p}_N} \ p( \mathcal{Z} \ | \ \mathbf{p}_1, \ldots, \mathbf{p}_N) \\
    &= \argmax_{\mathbf{p}_1, \ldots, \mathbf{p}_N} \prod_{j=1}^M p(\mathbf{z}_j \ | \ \mathbf{p}_1, \ldots, \mathbf{p}_N) \\
    &= \argmax_{\mathbf{p}_1, \ldots, \mathbf{p}_N} \prod_{j=1}^M \exp \left( -(\mathbf{e}_j(\mathbf{z}_j, \hat{\mathbf{z}}_j))^\transp \Omega_j \mathbf{e}_j(\mathbf{z}_j, \hat{\mathbf{z}}_j) \right) \\
    &= \argmin_{\mathbf{p}_1, \ldots, \mathbf{p}_N} \sum_{j=1}^M (\mathbf{e}_j(\mathbf{z}_j, \hat{\mathbf{z}}_j))^\transp \Omega_j \mathbf{e}_j(\mathbf{z}_j, \hat{\mathbf{z}}_j).
\end{align*}
%
We define
%
\begin{equation*}
    \chi^2 := \sum_{j=1}^M (\mathbf{e}_j(\mathbf{z}_j, \hat{\mathbf{z}}_j))^\transp \Omega_j \mathbf{e}_j(\mathbf{z}_j, \hat{\mathbf{z}}_j),
\end{equation*}
%
and this is what we seek to minimize.


\section{Dimensionality and Pose Representation}

Before proceeding further, it is helpful to discuss the dimensionality of the problem.  We have:
\begin{itemize}
  \item A set of $N$ poses $\mathbf{p}_1, \mathbf{p}_2, \ldots, \mathbf{p}_N$, where each pose lies on the manifold $\mathcal{M}$
  \begin{itemize}
    \item Each pose $\mathbf{p}_i$ is represented as a vector in (a subset of) $\mathbb{R}^d$.  For example:
    \begin{itemize}
      \item[$\circ$] An $SE(2)$ pose is typically represented as $(x, y, \theta)$, and thus $d = 3$.
      \item[$\circ$] An $SE(3)$ pose is typically represented as $(x, y, z, q_x, q_y, q_z, q_w)$, where $(x, y, z)$ is a point in $\mathbb{R}^3$ and $(q_x, q_y, q_z, q_w)$ is a \keyterm{quaternion}, and so $d = 7$.
    \end{itemize}
    \item We also need to be able to represent each pose compactly as a vector in (a subset of) $\mathbb{R}^c$.
    \begin{itemize}
      \item[$\circ$] Since an $SE(2)$ pose has three degrees of freedom, the $(x, y, \theta)$ representation is again sufficient and $c=3$.  
      \item[$\circ$] An $SE(3)$ pose only has six degrees of freedom, and we can represent it compactly as $(x, y, z, q_x, q_y, q_z)$, and thus $c=6$.
    \end{itemize}
    \item We use the $\boxplus$ operator to indicate pose composition when one or both of the poses are represented compactly.  The output can be a pose in $\mathcal{M}$ or a vector in $\mathbb{R}^c$, as required by context.
  \end{itemize}
  \item A set of $M$ measurements $\mathcal{Z} = \{\mathbf{z}_1, \mathbf{z}_2, \ldots, \mathbf{z}_M\}$
  \begin{itemize}
    \item Each measurement's dimensionality can be unique, and we will use $\bullet$ to denote a ``wildcard'' variable.
    \item Measurement $\mathbf{z}_j \in \mathbb{R}^\bullet$ has an associated information matrix $\Omega_j \in \mathbb{R}^{\bullet \times \bullet}$ and residual function $\mathbf{e}_j(\mathbf{z}_j, \hat{\mathbf{z}}_j) = \mathbf{e}_j(\mathbf{z}_j, \mathbf{p}_1, \ldots, \mathbf{p}_N) \in \mathbb{R}^\bullet$.
    \item A measurement could, in theory, constrain anywhere from 1 pose to all $N$ poses.  In practice, each measurement usually constrains only 1 or 2 poses.  
  \end{itemize}
\end{itemize}


\section{Graph SLAM Algorithm}

The ``Graph'' in Graph SLAM refers to the fact that we view the problem as a graph.  The graph has a set $\mathcal{V}$ of $N$ vertices, where each vertex $v_i$ has an associated pose $\mathbf{p}_i$.  Similarly, the graph has a set $\mathcal{E}$ of $M$ edges, where each edge $e_j$ has an associated measurement $\mathbf{z}_j$.  In practice, the edges in this graph are either unary (i.e., a loop) or binary.  (Note: $e_j$ refers to the edge in the graph associated with measurement $\mathbf{z}_j$, whereas $\mathbf{e}_j$ refers to the residual function associated with $\mathbf{z}_j$.)

We want to optimize
%
\begin{equation*}
    \chi^2 = \sum_{e_j \in \mathcal{E}} \mathbf{e}_j^\transp \Omega_j \mathbf{e}_j.
\end{equation*}
%
Let $\mathbf{x}_i \in \mathbb{R}^c$ be the compact representation of pose $\mathbf{p}_i \in \mathcal{M}$, and let
%
\begin{equation*}
    \mathbf{x} := \begin{bmatrix} \mathbf{x}_1 \\ \mathbf{x}_2 \\ \vdots \\ \mathbf{x}_N \end{bmatrix} \in \mathbb{R}^{cN}
\end{equation*}
%
We will solve this optimization problem iteratively.  Let
%
\begin{equation}
    \mathbf{x}^{k+1} := \mathbf{x}^k \boxplus \Delta \mathbf{x}^k = \begin{bmatrix} \mathbf{x}_1 \boxplus \Delta \mathbf{x}_1 \\ \mathbf{x}_2 \boxplus \Delta \mathbf{x}_2 \\ \vdots \\ \mathbf{x}_N \boxplus \Delta \mathbf{x}_2 \end{bmatrix} \label{eq:update}
\end{equation}
%
The $\chi^2$ error at iteration $k+1$ is
\begin{equation}
    \chi_{k+1}^2 = \sum_{e_j \in \mathcal{E}} \underbrace{\left[ \mathbf{e}_j(\mathbf{x}^{k+1}) \right]^\transp}_{1 \times \bullet} \underbrace{\Omega_j}_{\bullet \times \bullet} \underbrace{\mathbf{e}_j(\mathbf{x}^{k+1})}_{\bullet \times 1}.  \label{eq:chisq_at_kplusone}
\end{equation}
%
We will linearize the residuals as:
%
\begin{align}
    \mathbf{e}_j(\mathbf{x}^{k+1}) &= \mathbf{e}_j(\mathbf{x}^k \boxplus \Delta \mathbf{x}^k) \notag \\
    &\approx \mathbf{e}_j(\mathbf{x}^{k}) + \frac{\partial}{\partial \Delta \mathbf{x}^k} \left[ \mathbf{e}_j(\mathbf{x}^k \boxplus \Delta \mathbf{x}^k) \right] \Delta \mathbf{x}^k \notag \\
    &= \mathbf{e}_j(\mathbf{x}^{k}) + \left( \left. \frac{\partial \mathbf{e}_j(\mathbf{x}^k \boxplus \Delta \mathbf{x}^k)}{\partial (\mathbf{x}^k \boxplus \Delta \mathbf{x}^k)} \right|_{\Delta \mathbf{x}^k = \mathbf{0}} \right) \frac{\partial (\mathbf{x}^k \boxplus \Delta \mathbf{x}^k)}{\partial \Delta \mathbf{x}^k} \Delta \mathbf{x}^k.  \label{eq:linearization}
\end{align}
%
Plugging \eqref{eq:linearization} into \eqref{eq:chisq_at_kplusone}, we get:
%
\small
\begin{align}
    \chi_{k+1}^2 &\approx \ \ \ \ \ \sum_{e_j \in \mathcal{E}} \underbrace{[ \mathbf{e}_j(\mathbf{x}^k)]^\transp}_{1 \times \bullet} \underbrace{\Omega_j}_{\bullet \times \bullet} \underbrace{\mathbf{e}_j(\mathbf{x}^k)}_{\bullet \times 1} \notag \\
    &\hphantom{\approx} \ \ \ + \sum_{e_j \in \mathcal{E}} \underbrace{[ \mathbf{e}_j(\mathbf{x^k}) ]^\transp }_{1 \times \bullet} \underbrace{\Omega_j}_{\bullet \times \bullet} \underbrace{\left( \left. \frac{\partial \mathbf{e}_j(\mathbf{x}^k \boxplus \Delta \mathbf{x}^k)}{\partial (\mathbf{x}^k \boxplus \Delta \mathbf{x}^k)} \right|_{\Delta \mathbf{x}^k = \mathbf{0}} \right)}_{\bullet \times dN} \underbrace{\frac{\partial (\mathbf{x}^k \boxplus \Delta \mathbf{x}^k)}{\partial \Delta \mathbf{x}^k}}_{dN \times cN} \underbrace{\Delta \mathbf{x}^k}_{cN \times 1} \notag \\
    &\hphantom{\approx} \ \ \ + \sum_{e_j \in \mathcal{E}} \underbrace{(\Delta \mathbf{x}^k)^\transp}_{1 \times cN} \underbrace{ \left( \frac{\partial (\mathbf{x}^k \boxplus \Delta \mathbf{x}^k)}{\partial \Delta \mathbf{x}^k} \right)^\transp}_{cN \times dN} \underbrace{\left( \left. \frac{\partial \mathbf{e}_j(\mathbf{x}^k \boxplus \Delta \mathbf{x}^k)}{\partial (\mathbf{x}^k \boxplus \Delta \mathbf{x}^k)} \right|_{\Delta \mathbf{x}^k = \mathbf{0}} \right)^\transp}_{dN \times \bullet} \underbrace{\Omega_j}_{\bullet \times \bullet} \underbrace{\left( \left. \frac{\partial \mathbf{e}_j(\mathbf{x}^k \boxplus \Delta \mathbf{x}^k)}{\partial (\mathbf{x}^k \boxplus \Delta \mathbf{x}^k)} \right|_{\Delta \mathbf{x}^k = \mathbf{0}} \right)}_{\bullet \times dN} \underbrace{\frac{\partial (\mathbf{x}^k \boxplus \Delta \mathbf{x}^k)}{\partial \Delta \mathbf{x}^k}}_{dN \times cN} \underbrace{\Delta \mathbf{x}^k}_{cN \times 1} \notag \\
    &= \chi_k^2 + 2 \mathbf{b}^\transp \Delta \mathbf{x}^k + (\Delta \mathbf{x}^k)^\transp H \Delta \mathbf{x}^k,  \notag
\end{align}
\normalsize
%
where
%
\begin{align*}
    \mathbf{b}^\transp &= \sum_{e_j \in \mathcal{E}} \underbrace{[ \mathbf{e}_j(\mathbf{x^k}) ]^\transp }_{1 \times \bullet} \underbrace{\Omega_j}_{\bullet \times \bullet} \underbrace{\left( \left. \frac{\partial \mathbf{e}_j(\mathbf{x}^k \boxplus \Delta \mathbf{x}^k)}{\partial (\mathbf{x}^k \boxplus \Delta \mathbf{x}^k)} \right|_{\Delta \mathbf{x}^k = \mathbf{0}} \right)}_{\bullet \times dN} \underbrace{\frac{\partial (\mathbf{x}^k \boxplus \Delta \mathbf{x}^k)}{\partial \Delta \mathbf{x}^k}}_{dN \times cN} \\
    H &= \sum_{e_j \in \mathcal{E}} \underbrace{ \left( \frac{\partial (\mathbf{x}^k \boxplus \Delta \mathbf{x}^k)}{\partial \Delta \mathbf{x}^k} \right)^\transp}_{cN \times dN} \underbrace{\left( \left. \frac{\partial \mathbf{e}_j(\mathbf{x}^k \boxplus \Delta \mathbf{x}^k)}{\partial (\mathbf{x}^k \boxplus \Delta \mathbf{x}^k)} \right|_{\Delta \mathbf{x}^k = \mathbf{0}} \right)^\transp}_{dN \times \bullet} \underbrace{\Omega_j}_{\bullet \times \bullet} \underbrace{\left( \left. \frac{\partial \mathbf{e}_j(\mathbf{x}^k \boxplus \Delta \mathbf{x}^k)}{\partial (\mathbf{x}^k \boxplus \Delta \mathbf{x}^k)} \right|_{\Delta \mathbf{x}^k = \mathbf{0}} \right)}_{\bullet \times dN} \underbrace{\frac{\partial (\mathbf{x}^k \boxplus \Delta \mathbf{x}^k)}{\partial \Delta \mathbf{x}^k}}_{dN \times cN}.
\end{align*}
%
Using this notation, we obtain the optimal update as
%
\begin{equation}
    \Delta \mathbf{x}^k = -H^{-1} \mathbf{b}.  \label{eq:deltax}
\end{equation}
%
We apply this update to the poses via \eqref{eq:update} and repeat until convergence.



\end{document}
